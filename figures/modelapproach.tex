\documentclass[class=minimal,border=2pt]{standalone}

\usepackage{tikz}
\usetikzlibrary{shapes,arrows}
\usepackage{verbatim}

\begin{document}
	
	
	\tikzstyle{block} = [draw, fill=blue!20, rectangle, 
	minimum height=3em, minimum width=6em]
	\tikzstyle{sum} = [draw, fill=blue!20, circle, node distance=1cm]
	\tikzstyle{input} = [coordinate]
	\tikzstyle{output} = [coordinate]
	\tikzstyle{pinstyle} = [pin edge={to-,thin,black}]
	
	% The block diagram code is probably more verbose than necessary
	\begin{tikzpicture}[auto, node distance=2cm,>=latex']
	% We start by placing the blocks
	\node [input, name=input] {};
	\node [block, right of=input] (sum) {Independent $\mathbf{x}$};
	\node [block, right of=sum, node distance=3.5cm] (controller) {Theoretical model};
	\node [block, right of=controller, node distance=3.3cm] (system) {Dependent $\mathbf{y}$};
	% We draw an edge between the controller and system block to 
	% calculate the coordinate u. We need it to place the measurement block. 
	\draw [<-, thick] (controller) -- node[name=u] {} (system);
	\node [output, right of=system] (output) {};
	\node [block, below of=controller] (measurements) {Test of hypotheses on $\mathbf{\beta}$};
	
	% Once the nodes are placed, connecting them is easy. 
	%\draw [draw,->] (input) -- node {$r$} (sum);
	\draw [->, thick] (sum) -- node {} (controller);
	\draw [->, thick] (measurements) -- node {$H_{\beta_0}$} (controller);
	\end{tikzpicture}
\end{document}