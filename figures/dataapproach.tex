\documentclass[class=minimal,border=2pt]{standalone}

\usepackage{tikz}
\usetikzlibrary{shapes,arrows}
\usepackage{verbatim}

\begin{comment}
:Title: Control system principles
:Tags: Block diagrams 

An example of a control system with a feedback loop. Block diagrams like this
are quite time consuming to create by hand. The relative node placement feature
makes it a bit easier, but it works best when the nodes have equal widths. 
However, the results are quite pleasing and hopefully worth the effort. 
You can probably speed up the process by creating custom macros. 
\end{comment}

\begin{document}
	
	
	\tikzstyle{block} = [draw, fill=blue!20, rectangle, 
	minimum height=3em, minimum width=6em]
	\tikzstyle{sum} = [draw, fill=blue!20, circle, node distance=1cm]
	\tikzstyle{input} = [coordinate]
	\tikzstyle{output} = [coordinate]
	
	% The block diagram code is probably more verbose than necessary
	\begin{tikzpicture}[auto, node distance=2cm,>=latex']
	% We start by placing the blocks
	\node [input, name=input] {};
	\node [block, right of=input] (sum) {Predictors $\mathbf{x}$};
	\node [block, right of=sum, node distance=3cm] (controller) {Unknown model};
	\node [block, right of=controller, node distance=3cm] (system) {Response $\mathbf{y}$};
	% We draw an edge between the controller and system block to 
	% calculate the coordinate u. We need it to place the measurement block. 
	\draw [->, thick] (controller) -- node[name=u] {} (system);
	\node [output, right of=system] (output) {};
	\node [block, below of=u] (measurements) {Model validation};
	
	% Once the nodes are placed, connecting them is easy. 
	%\draw [draw,->] (input) -- node {$r$} (sum);
	\draw [->, thick] (sum) -- node {} (controller);
	\draw [-, thick] (system) -- node [name=y] {}(output);
	\draw [->, thick] (output) |- (measurements);
	\draw [->, thick] (measurements) -|  (controller);
	\end{tikzpicture}
\end{document}